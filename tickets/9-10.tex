
\documentclass[a4paper]{article}

\usepackage{fontspec}
\usepackage{iftex}
\ifXeTeX{}
   \setmainfont{cmun}[
    Extension=.otf,UprightFont=*rm,ItalicFont=*ti,
    BoldFont=*bx,BoldItalicFont=*bi,
   ]
   \setsansfont{cmun}[
    Extension=.otf,UprightFont=*ss,ItalicFont=*si,
    BoldFont=*sx,BoldItalicFont=*so,
   ]
   \setmonofont{cmun}[
    Extension=.otf,UprightFont=*btl,ItalicFont=*bto,
    BoldFont=*tb,BoldItalicFont=*tx,
   ]
\else
   \setmainfont{CMU Serif}
   \setsansfont{CMU Sans Serif}
   \setmonofont{CMU Typewriter Text}
\fi
\defaultfontfeatures{Ligatures=TeX}
\usepackage{polyglossia}
\usepackage[autostyle=true]{csquotes}
\setdefaultlanguage{russian}
\setotherlanguage{english}
\frenchspacing{}

\usepackage{amsmath}
\usepackage{amsthm}
\usepackage{amsfonts}
\usepackage{amssymb}
\usepackage[colorlinks]{hyperref}
\usepackage{gauss}
\usepackage{tikz}

\theoremstyle{definition}
\newtheorem{definition}{Определение}
\theoremstyle{plain}
\newtheorem{theorem}{Теорема}
\newtheorem{lemma}{Лемма}

\author{Александр Старовойтов %
$<$\href{https://t.me/stewkk}%
{Telegram}$>$}
\title{Билеты 9-10}

\begin{document}
\maketitle
\section*{Билет 9}
\subsection*{Задание 1}
\begin{itemize}
  \item
Дать определения векторного и смешанного произведений векторов. (вопрос 10)
  \item
Вывести формулы для вычисления векторного произведения и смешанного произведения векторов в правом ортонормированном базисе. (вопрос 11)
\end{itemize}
\subsection*{Задание 2}
$|\vec{a}| = 3; |\vec{b}| = \sqrt{2}; |\vec{c}| = 4$\\
$( \widehat{\vec{a}, \vec{b}} ) = ( \widehat{\vec{b}, \vec{c}} ) = 45^{\circ}; ( \widehat{\vec{a}, \vec{c}} ) = 60^{\circ}$\\
$-\vec{a} + 2 \vec{b} + \vec{c} =$ ?\\

$\vec{d}^{2}=(-\vec{a} + 2 \vec{b} + \vec{c}, -\vec{a} + 2 \vec{b} + \vec{c}) = \vec{a}^{2} -2 \vec{a} \vec{b} - \vec{a} \vec{c} - 2 \vec{a} \vec{b} + 4 \vec{b}^{2} + 2 \vec{b} \vec{c} - \vec{a} \vec{c} + 2 \vec{b} \vec{c} + \vec{c}^{2} =$\\
$9 + 8 + 16 - 4 \vec{a}\vec{b} - 2 \vec{a}\vec{c} + 4 \vec{b}\vec{c} = 33 - 12 - 12 + 16 = 25 \Rightarrow$ \fbox{$|\vec{d}|$ = 5}
\subsection*{Задание 3}
\emph{Составить каноническое уравнение процекции прямой $L$ на \mbox{плоскость $\alpha$}}\\
$L: \begin{cases}
  2x + 3y = 5\\
  z = 1
\end{cases}$\\
$\alpha: 2x+3y+6z = 6$

$\vec{q}_{L}\{3, -2, 0\}; \vec{n}_{\alpha}\{2, 3, 6\}$\\
Пр$_{\alpha}\vec{q}_{L} = \vec{q}_{L} - \frac{(\vec{q}_{L}, \vec{n}_{\alpha})}{(\vec{n}_{\alpha}, \vec{n}_{\alpha})} \vec{n}_{\alpha} = %
\{3, -2, 0\} - \frac{6 - 6}{49}\{2, 3, 6\} = \{3, -2, 0\} = \vec{q}_{L}$\\
$M = (1, 1, 1) \in L; A = (0, 0, 1) \in \alpha; AM\{1, 1, 0\}$\\
Найдем проекцию т. $M$ на $\alpha$:\\
Пр$_{\alpha}\vec{AM} = \vec{AM} - \frac{(\vec{AM}, \vec{n}_{\alpha})}{(\vec{n}_{\alpha}, \vec{n}_{\alpha})}\vec{n}_{\alpha} = %
\begin{pmatrix}
  1\\ 1\\ 0
\end{pmatrix}
-
\frac{5}{49}
\begin{pmatrix}
  2\\ 3\\ 6
\end{pmatrix}
=
\begin{pmatrix}
  39/49\\ 34/49\\ -30/49
\end{pmatrix}
\Rightarrow$\\
$\Rightarrow$ Пр$_{\alpha}M = (39/49, 34/49, 19/49)$\\
Тогда уравнение проекции: \fbox{$\frac{x - 39/49}{3} = \frac{y - 34/49}{-2} = \frac{z - 19/49}{0}$}

\subsection*{Задание 4}

\emph{Найти \emph{НОД} многочленов $f$ и $g$ с коэффициентами в поле вычетов $\mathbb{Z}_{5}$}
\[
f = x^{5} + 3x^{4} + 2x^{3} + x^{2} + 3x + 2
\]
\[
g = x^{4} + 2x^{3} + 3x + 1
\]

\noindent $
x^{5} + 3x^{4} + 2x^{3} + x^{2} + 3x + 2
=
(x^{4} + 2x^{3} + 3x + 1)
(x + 1)
+
(3x^{2}+4x+1)
$\\
$
2(3x^{2}+4x+1) = x^{2}+3x+2
$\\
$
x^{4} + 2x^{3} + 3x + 1
=
(x^{2}+3x+2)
(x^{2}+4x+1)
+
(2x+4)
$\\
$
3(2x+4) = x+2
$\\
$
x^{2}+3x+2
=
(x + 2)
(x + 1)
$\\
Последний ненулевой остаток $x + 2 \Rightarrow$ \fbox{НОД$(f, g) = x + 2$}
\subsection*{Задание 5}
\emph{Найти все матрицы перестановочные с матрицей $A$}\\
$A =
\begin{pmatrix}
  1 & 2\\
  3 & 1
\end{pmatrix}$

$AB = BA$\\
$B =
\begin{pmatrix}
  x_{1} & x_{2}\\
  x_{3} & x_{4}
\end{pmatrix}$\\
$AB =
\begin{pmatrix}
  1 & 2\\
  3 & 1
\end{pmatrix}
\begin{pmatrix}
  x_{1} & x_{2}\\
  x_{3} & x_{4}
\end{pmatrix}
=
\begin{pmatrix}
  x_{1}+2x_{3} & x_{2} + 2x_{4}\\
  3x_{1} + x_{3} & 3x_{2} + x_{4}
\end{pmatrix}
$\\
$BA =
\begin{pmatrix}
  x_{1} & x_{2}\\
  x_{3} & x_{4}
\end{pmatrix}
\begin{pmatrix}
  1 & 2\\
  3 & 1
\end{pmatrix}
=
\begin{pmatrix}
  x_{1}+3x_{2} & 2x_{1} + x_{2}\\
  x_{3} + 3x_{4} & 2x_{3} + x_{4}
\end{pmatrix}
$\\
$
\begin{cases}
  x_{1} + 2x_{3} = x_{1} + 3x_{2}\\
  x_{2} + 2x_{4} = 2x_{1} + x_{2}\\
  3x_{1} + x_{3} = x_{3} + 3x_{4}\\
  3x_{2} + x_{4} = 2x_{3} + x_{4}
\end{cases}
$\\
$
\begin{cases}
  -3x_{2} + 2x_{3} = 0\\
  -2x_{1} + 2x_{4} = 0\\
  3x_{1} - 3x_{4} = 0\\
  3x_{2} - 2x_{3} = 0
\end{cases}
$\\
\begin{multline*}
\begin{gmatrix}[p]
  0 & -3 & 2 & 0\\
  -2 & 0 & 0 & 2\\
  3 & 0 & 0 & -3\\
  0 & 3 & -2 & 0
  \rowops
  \mult{2}{: 3}
\end{gmatrix}
=
\begin{gmatrix}[p]
  0 & -3 & 2 & 0\\
  -2 & 0 & 0 & 2\\
  1 & 0 & 0 & -1
\end{gmatrix}
=
\begin{gmatrix}[p]
  0 & -3 & 2 & 0\\
  1 & 0 & 0 & -1
  \rowops
\end{gmatrix}
\end{multline*}\\
$
x_{3} = c_{1}; x_{4} = c_{2}
$\\
$
x_{1} - x_{4} = 0
$\\
$
-3x_{2} + 2x_{3} = 0
$\\
$
x_{1} = x_{4} = c_{2}
$\\
$
x_{2} = \frac{2}{3}x_{3} = \frac{2}{3}c_{1}
$\\
$
B =
\begin{pmatrix}
  x_{1} & x_{2}\\
  x_{3} & x_{4}
\end{pmatrix}
=
$
\fbox{
$
\begin{pmatrix}
  c_{2} & \frac{2}{3}c_{1}\\
  c_{1} & c_{2}
\end{pmatrix}
$
}

\section*{Билет 10}
\subsection*{Задание 1}
\begin{itemize}
\item
Сформулировать теорему об уравнении первого порядка как уравнении прямой на плоскости.
(вопрос 12)
\item
Дать определение пучка прямых на плоскости (собственного и несобственного).
Доказать теорему об уравнении собственного пучка прямых.
(вопрос 14)
\end{itemize}
\subsection*{Задание 2}
$ABCD$ --- тетраэдр\\
$V_{ABCD} = 7; A = (-2,1,0); B = (0, -2, 1); C = (1, 0, -2)$\\
$D = (x, y, z), y < 0$

$\vec{AB}\{2, -3, 1\}; \vec{AC}\{3, -1, -2\}; \vec{AD}\{x + 2, y - 1, z\}$\\
$V_{ABCD} = \frac{\vec{AB}\vec{AC}\vec{AD}}{6} \Rightarrow \vec{AB}\vec{AC}\vec{AD} = 42$\\
$\begin{vmatrix}
  2 & -3 & 1\\
  3 & -1 & -2\\
  x+2 & y-1 & z
\end{vmatrix}
= 42$\\
$-2z + 3y - 3 + 6x + 12 + x + 2 + 9z + 4y - 4 = 42$\\
$7x + 7y + 7z = 35$\\
\fbox{$D = (6, -1, 0)$}
\subsection*{Задание 3}

\emph{Уравнение кривой второго порядка привести к каноническому виду.
Определить основные параметры (полуоси, координаты центра, фокусов, эксцентриситет) и сделать чертеж кривой в исходной системе координат.}
\[
4x^{2}+3y^{2}-8x+12y-32=0
\]

$4(x^{2}-2x+1) + 3(y^{2}+4y+4) - 48 = 4(x-1){}^{2} + 3(y + 2){}^{2} - 48$\\
$\frac{(x-1){}^{2}}{12} + \frac{(y+2){}^{2}}{16} = 1$\\
$x' = x - 1; y' = y + 2$\\
$O(1, -2)$\\
$c = \sqrt{b^{2}-a^{2}} = 2$\\
$a = 2\sqrt{3}; b = 4$\\
$F_{1}(1, 0); F_{2}(1, -4)$\\
$e = \frac{c}{a} = \frac{2}{2\sqrt{3}}=\frac{\sqrt{3}}{3}$\\
\begin{tikzpicture}[domain=0:1]
\draw[thick,color=gray,
dashed] (-3, -6) grid (5,2);
\draw[->] (-3.2,0) -- (5.2,0)
node[below right] {$x$};
\draw[->] (0,-6.2) -- (0,2.2)
node[left] {$y$};
\draw (1,-2) ellipse (3.46 and 4);
\end{tikzpicture}
\subsection*{Задание 4}

\emph{Является ли неприводимым многочлен $f$ с коэффициентами в поле вычетов $\mathbb{Z}_{3}$.
Если нет, разложить его на неприводимые множители}
\[
f = x^{4} + x^{3} + x + 2
\]

Проверим есть ли в разложении линейные двучлены:\\
$f(0) = 2 \neq 0; f(1) = 5 = 2 \neq 0; f(2) = 1 + 2 + 2 + 2 = 1 \neq 0$\\
Квадратные трехчлены (перебором):\\
$
x^{4}+x^{3}+x+2
=
(x^{2} + 1)
(x^{2}+x+2)
$

\subsection*{Задание 5}
\emph{Решить матричное уравнение}

\begin{equation*}
  X
  \begin{pmatrix}
    0 & 1 & 1\\
    1 & 0 & 1\\
    1 & 1 & 0
  \end{pmatrix}
  =
  \begin{pmatrix}
    -2 & -1 & -1\\
    -1 & 0 & 1\\
    1 & 1 & 2
  \end{pmatrix}
\end{equation*}
\begin{equation*}
  X
  =
  \begin{pmatrix}
    -2 & -1 & -1\\
    -1 & 0 & 1\\
    1 & 1 & 2
  \end{pmatrix}
  \begin{pmatrix}
    0 & 1 & 1\\
    1 & 0 & 1\\
    1 & 1 & 0
  \end{pmatrix}^{-1}
\end{equation*}
\begin{equation}\label{eq:1}
  \begin{pmatrix}
    0 & 1 & 1\\
    1 & 0 & 1\\
    1 & 1 & 0
  \end{pmatrix}^{-1}
  =
  \frac{1}{\det A}A^{V} =
\end{equation}
\begin{equation*}
  \det A =
  \begin{vmatrix}
    0 & 1 & 1\\
    1 & 0 & 1\\
    1 & 1 & 0
  \end{vmatrix}
  =
  1 + 1
  =
  2
\end{equation*}
\begin{equation*}
  A^{(1,1)}=
  \begin{vmatrix}
    0 & 1\\
    1 & 0
  \end{vmatrix}
  = -1
  , \qquad
  A^{(1, 2)} = -
  \begin{vmatrix}
    1 & 1\\
    1 & 0
  \end{vmatrix}
  = 1, \qquad
  A^{(1, 3)} =
  \begin{vmatrix}
    1 & 0\\
    1 & 1
  \end{vmatrix}
  = 1
\end{equation*}
\begin{equation*}
  A^{(2, 1)} = -
  \begin{vmatrix}
    1 & 1\\
    1 & 0
  \end{vmatrix}
  = 1, \qquad
  A^{(2, 2)} =
  \begin{vmatrix}
    0 & 1\\
    1 & 0
  \end{vmatrix}
  = -1, \qquad
  A^{(2, 3)} = -
  \begin{vmatrix}
    0 & 1\\
    1 & 1
  \end{vmatrix}
  = 1
\end{equation*}
\begin{equation*}
  A^{(3, 1)} =
  \begin{vmatrix}
    1 & 1\\
    0 & 1
  \end{vmatrix}
  = 1, \qquad
  A^{(3, 2)} = -
  \begin{vmatrix}
    0 & 1\\
    1 & 1
  \end{vmatrix}
  = 1, \qquad
  A^{(3, 3)} =
  \begin{vmatrix}
    0 & 1\\
    1 & 0
  \end{vmatrix}
  = -1
\end{equation*}
\begin{equation*}
  A^{V} =
  \begin{pmatrix}
    -1 & 1 & 1\\
    1 & -1 & 1\\
    1 & 1 & -1
  \end{pmatrix}^{T}
  =
  \begin{pmatrix}
    -1 & 1 & 1\\
    1 & -1 & 1\\
    1 & 1 & -1
  \end{pmatrix}
\end{equation*}
\begin{equation*}
  (\ref{eq:1}) =
  \frac{1}{2}
  \begin{pmatrix}
    -1 & 1 & 1\\
    1 & -1 & 1\\
    1 & 1 & -1
  \end{pmatrix}
  =
  \begin{pmatrix}
    -1/2 & 1/2 & 1/2\\
    1/2 & -1/2 & 1/2\\
    1/2 & 1/2 & -1/2
  \end{pmatrix}
\end{equation*}
\begin{multline*}
  X =
  \begin{pmatrix}
    -2 & -1 & -1\\
    -1 & 0 & 1\\
    1 & 1 & 2
  \end{pmatrix}
  \begin{pmatrix}
    -1/2 & 1/2 & 1/2\\
    1/2 & -1/2 & 1/2\\
    1/2 & 1/2 & -1/2
  \end{pmatrix}
  =\\
  =
  \begin{pmatrix}
    1 - 1/2 - 1/2 & -1 + 1/2 - 1/2 & -1 - 1/2 + 1/2\\
    -1/2 + 0 + 1/2 & -1/2 + 0 + 1/2 & -1/2 + 0 - 1/2\\
    -1/2 + 1/2 + 1 & 1/2 - 1/2 + 1 & 1/2 + 1/2 - 1
  \end{pmatrix}
  \fbox{$=
    \begin{pmatrix} 0 & -1 & -1\\
      0 & 0 & -1\\\
      1 & 1 & 0
    \end{pmatrix}$}
\end{multline*}
\end{document}

\documentclass[a4paper]{article}

\usepackage{fontspec}
\usepackage{iftex}
\ifXeTeX{}
   \setmainfont{cmun}[
    Extension=.otf,UprightFont=*rm,ItalicFont=*ti,
    BoldFont=*bx,BoldItalicFont=*bi,
   ]
   \setsansfont{cmun}[
    Extension=.otf,UprightFont=*ss,ItalicFont=*si,
    BoldFont=*sx,BoldItalicFont=*so,
   ]
   \setmonofont{cmun}[
    Extension=.otf,UprightFont=*btl,ItalicFont=*bto,
    BoldFont=*tb,BoldItalicFont=*tx,
   ]
\else
   \setmainfont{CMU Serif}
   \setsansfont{CMU Sans Serif}
   \setmonofont{CMU Typewriter Text}
\fi
\defaultfontfeatures{Ligatures=TeX}
\usepackage{polyglossia}
\usepackage[autostyle=true]{csquotes}
\setdefaultlanguage{russian}
\setotherlanguage{english}
\frenchspacing{}

\usepackage{amsmath}
\usepackage{amsthm}
\usepackage{amsfonts}
\usepackage{amssymb}
\usepackage[colorlinks]{hyperref}

\theoremstyle{definition}
\newtheorem{definition}{Определение}
\theoremstyle{plain}
\newtheorem{theorem}{Теорема}
\newtheorem{lemma}{Лемма}

\author{Александр Старовойтов %
$<$\href{https://t.me/stewkk}%
{Telegram}$>$}
\title{Ответы на вопросы 36-38}

\begin{document}
\maketitle
\section{Вопрос 36}
\textbf{Дать определение группы подстановок.
Сформулировать и доказать теорему о разложении подстановки в произведение независимых циклов и произведение транспозиций.}

\begin{definition}[Группа подстановок (или группа перестановок)]
  Это множество всех биекций n-элементного множества на себя с операцией композиции биекций.
  Множество всех перестановок множества $\{1, 2, \ldots{} n\}$ обозначается $S_{n}$.
\end{definition}
Элемент этой группы называется подстановкой:
\begin{equation*}
  \pi =
  \begin{pmatrix}
    1 & 2 & \ldots & n\\
    i_{1} & i_{2} & \ldots & i_{n}
  \end{pmatrix}
\end{equation*}
$\pi(1) = i_{1}, \pi(2) = i_{2} \ldots$

Операцию композиции будем называть произведением $(\pi\circ\rho)(x) = \rho(\pi(x))$.

Через $\begin{pmatrix}i_{1}&i_{2}&\ldots&i_{n}\end{pmatrix}$ обозначается цикл длины n, который переводит $i_{1} \mapsto i_{2}$, $i_{2} \mapsto i_{3}$, \ldots, $i_{n} \mapsto i_{1}$.
Циклы являются независимыми, если они представляют непересекающиеся множества элементов.

Транспозицией называется цикл длины 2.

\begin{theorem}[О разложении подстановки в произведение независимых циклов и произведение транспозиций]
  a) Любая перестановка разбивается в произведение непересекающихся циклов.
  б) Любая перестановка может быть представлена в виде произведения транспозиций.
\end{theorem}
\begin{proof}
  а) Пусть $\pi \in S_{n}$. Если \forall i \pi(i)=i, то \pi{} = e.

  Пусть $\exists i: \pi(i) \neq i$ и $i_{1}$ --- первый такой элемент, что $\pi(i_{1}) \neq i_{1} \Rightarrow \pi(i_{1}) > i_{1}$\\
  Тогда пусть $\pi(i_{1}) = i_{2}$; Ясно, что $\pi(i_{2}) \neq i_{2}$ (иначе перестановка не взаимооднозначна) $\Rightarrow \pi(i_{2}) = i_{3}$\\
  $|M| < \infty \Rightarrow \exists k \in \mathbb{N}: i_{k} \neq i_{1}, i_{k + 1} = \pi(i_{k}) = i_{1}$\\
  $\pi_{1} = \begin{pmatrix}i_{1}&i_{2}&\ldots&i_{k}\end{pmatrix}$\\
  Рассмотрим $\pi' = \pi \pi_{1}^{-1}$\\
  В $\pi'$ элементы $1, 2, \ldots{} i_{1}$ остаются на месте.
  Также остаются на месте элементы $i_{2}, \ldots, i_{k}$.
  Повторим рассуждение для $\pi'$.
  Получим цикл $\pi_{2}$, не пересекающийся с $\pi_{1}$; $\pi'' = \pi \pi_{1}^{-1} \pi_{2}^{-1}$, в $\pi''$ неподвижных элементов стало больше и т.д.

  После конечного числа шагов получим
  \begin{equation*}
    \pi \pi_{1}^{-1} \pi_{2}^{-1} \ldots \pi_{s}^{-1} = e
  \end{equation*}
  \begin{equation*}
    \pi = \pi_{s}\pi_{s-1} \ldots \pi_{1} \text{ --- требуемое разложение}
  \end{equation*}

  \emph{Единственность}:
  пусть $\pi = \pi_{1} \pi_{2} \ldots \pi_{s} = \tau_{1} \tau_{2} \ldots \tau_{r}$ --- два таких разложения.
  $i_{1}$ --- первый элемент, который не остается на месте: $\pi(i_{1}) \neq i_{1}$.
  Можно считать, что $i_{1}$ входит в $\pi_{1}$ и $\tau_{1}$.
  Далее легко видеть, что $\pi_{1} = \tau_{1} \Rightarrow \pi_{2} = \tau_{2}, \ldots, \pi_{s} = \tau_{r}$

  б) Ввиду а) достаточно разложить на транспозиции любой цикл.
  \begin{equation*}
    \begin{pmatrix}
      i_{1}&i_{2}&\ldots&i_{k}
    \end{pmatrix}
    =
    \begin{pmatrix}
      i_{1}&i_{2}
    \end{pmatrix}
    \begin{pmatrix}
      i_{1}&i_{3}
    \end{pmatrix}
    \ldots
    \begin{pmatrix}
      i_{1}&i_{k}
    \end{pmatrix}
  \end{equation*}
\end{proof}

\section{Вопрос 37}
\textbf{Дать определение чётной и нечётной подстановки.
Объяснить и обосновать, как чётность подстановки определяется по её разложению в произведение транспозиций.
Дать определение знакопеременной группы.}

\begin{definition}[Четная и нечетная подстановка]
  Подстановка $\pi \in S_{n}$ четная, если количество инверсий в $\pi$ четное, и нечетная в противном случае.\\
  $\pi \in S_{n}$, $1 \leq i < j \leq n$.
  Пара $(i, j)$ --- инверсия, если $\pi(i) > \pi(j)$.
\end{definition}

Подстановка $\pi \in S_{n}$ является четной $\Leftrightarrow$ количество транспозиций в ее разложении на транспозиции четно.

\begin{proof}
  Достаточно доказать, что если $\tau$ --- транспозиция, то подстановки $\pi$ и $\tau \pi$ имеют разную четность.

  Если $\tau = (i\ j) = (i\ i+1)(i+1\ i+2)\dots(j-1\ j)(j-2\ j-1)\ldots(i\ i+1)$.
  В этом разложении $j - i + j - i - 1 = 2 (j - i) - 1$ --- нечетное число транспозиций.\\
  $\Rightarrow$ можно считать, что $\tau$ переставляет соседние элементы.

  Пусть $\tau = (k\ k+1)$\\
  $\tau' = \tau\pi$
  \begin{equation*}
    \pi =
    \begin{pmatrix}
      1 & \ldots & k & & k + 1 & \ldots & n\\
      i_{1} & \ldots & i_{k} & & i_{k+1} & \ldots & i_{n}
    \end{pmatrix}
  \end{equation*}
  \begin{equation*}
    \pi' =
    \begin{pmatrix}
      1 & \ldots & k & & k + 1 & \ldots & n\\
      i_{1} & \ldots & i_{k+1} & & i_{k} & \ldots & i_{n}
    \end{pmatrix}
  \end{equation*}
  $\forall (l, m)$ --- инверсия в $\pi$\\
  $l \neq k, k+1, m \neq k, k+1 \Rightarrow (l, m)$ --- инверсия и в $\pi'$\\
  $l = k, m > k+1 \Rightarrow$ пара $(k+1, m)$ --- инверсия в $\pi'$\\
  $l = k+1 \Rightarrow (k, m)$ --- инверсия в $\pi'$\\
  $l \neq k,k+1, m = k$ или $k+1$ --- аналогично

  Пара $(k, k+1)$ правильная для $\pi \Leftrightarrow$ она инверсия для $\pi'$ $\Rightarrow i(\pi') = i(\pi) + 1$ т.е. четность разная.\\
  Аналогично, если $(k, k+1)$ - инверсия для $\pi$, то она правильная для $\pi'$ $\Rightarrow i(\pi') = i(\pi) - 1$ т.е. четность разная.
\end{proof}

\begin{definition}[Знакопеременная группа]
  Множество всех четных подстановок в группе $S_{n}$ образует подгруппу и называется знакопеременной группой на $n$ элементах.
  Обозначается $A_{n}$.
\end{definition}

\section{Вопрос 38}
\textbf{Дать определение кольца. Какое кольцо называется ассоциативным, коммутативным, кольцом с единицей?
Сформулировать и обосновать основные положения теории делимости в кольце целых чисел: бесконечность множества простых чисел, деление с остатком, наибольший общий делитель и алгоритм Евклида, основная теорема арифметики.}

\begin{definition}[Кольцо]
  Кольцо --- алгебраическая структура $(R,+,\cdot)$ с двумя бинарными операциями, удовлетворяющими аксиомам кольца:\\
  1. $(R, +)$ --- абелева группа;\\
  2. Дистрибутивность: $\forall a, b, c\ $ $a(b + c) = ab + ac$ и $(a + b)c = ac + bc$
\end{definition}
Если операция умножения ассоциативна, то R называют \emph{ассоциативным кольцом}; коммутативна --- \emph{коммутативным кольцом}.
Если существует нейтральный элемент по умножению, его обозначают 1, а кольцо называют \emph{кольцом с единицей}.

\begin{theorem}[О бесконечности множества простых чисел]
  Множество простых чисел бесконечно.
\end{theorem}
\begin{proof}
  Пусть $p_{1},\ldots,p_{k}$ --- все простые числа.
  Рассмотрим число $p = p_{1} \cdot p_{2} \cdot \ldots \cdot p_{k} + 1$.
  $p$ не делится ни на одно из $p_{1},\ldots,p_{k}$.
  Оно либо само простое, либо делится на простое $\neq p_{1},\ldots,p_{k}$. Противоречие.
\end{proof}

\begin{theorem}[О делении с остатком]
  Пусть $a, b \in \mathbb{Z}, b \neq 0$.
  Тогда $\exists! q, r \in \mathbb{Z}, 0 \leq r < |b|: a = qb+r$, где q --- (неполное) частное, а r --- остаток.
\end{theorem}
\begin{proof}
  Рассмотрим множество $M = \{a - k \cdot b\ |\ k \in \mathbb{Z}\}$.
  $r$ --- минимальное неотрицательное число множества $M$.

  \emph{Существование}:
  Покажем, что $r < |b|$:\\
  Пусть $r \geq |b|$: $r = a - kb$\\
  при $a \geq 0: r' = a - (k + 1)b$ \\
  при $a < 0: r' = a + (k - 1)b$\\
  $r' \in M, r' \geq 0, r' < r$ --- противоречие $\Rightarrow$ существование $q$ и $r$

  \emph{Единственность}:
  Пусть $a = q_{1}b + r_{1} = q_{2}b + r_{2}$;
  $0 \leq r_{1} < |b|$, $0 \leq r_{2} < |b|$\\
  Пусть $r_{2} \geq r_{1}$: $\underbrace{(q_{1} - q_{2})b}_{\geq |b|} = \underbrace{r_{2} - r_{1}}_{< |b|} \Rightarrow$ противоречие, если $q_{1} \neq q_{2}, r_{1} \neq r_{2}$
\end{proof}

\begin{definition}[НОД]
Число $d \in \mathbb{N}$ называют \emph{наибольшим общим делителем} чисел $a, b \in \mathbb{Z}$, если $d|a$, $d|b$, и $d$ --- наибольшее число с таким свойством ($a \neq 0$ или $b \neq 0$).
Обозначается $d = \text{НОД}(a, b) = gcd(a, b) = (a, b)$.
\end{definition}

\begin{definition}[Алгоритм Евклида]
  \begin{align*}
    &a, b \in \mathbb{Z}, b \neq 0\\
    a &= q_{1}b+r_{1}, 0 \leq r_{1} < |b|\\
    b &= q_{2}r_{1}+r_{2}, 0 \leq r_{2} < r_{1}\\
    &\cdots\\
    r_{k-2} &= q_{k}r_{k-1}+r_{k}, 0 \leq r_{n} < r_{k-1}\\
    r_{k-1} &= q_{k+1} r_{k}
  \end{align*}
  $r_{k}$ --- последний ненулевой остаток.
\end{definition}

\begin{theorem}[НОД и алгоритм Евклида]
  \begin{equation*}
    d = (a, b) = r_{k}
  \end{equation*}
\end{theorem}
\begin{proof}
  Пойдем по алгоритму (сверху вниз)
  $d | a, d | b \Rightarrow d | r_{1} \Rightarrow d | r_{2} \Rightarrow \ldots \Rightarrow d | r_{k}$\\
  (снизу вверх) $r_{k}|r_{k-1} \Rightarrow r_{k}|r_{k-2} \Rightarrow \ldots \Rightarrow r_{k}|b, r_{k}|a \Rightarrow r_{k} = d$
\end{proof}

\begin{lemma}[Следствие алгоритма Евклида]\label{lemma:1}
  Пусть $d = (a, b)$.
  Тогда $\exists u, v \in \mathbb{Z}, d = ua + vb$.
\end{lemma}
\begin{proof}
  $d = r_{k} = r_{k - 2} - q_{k} \cdot r_{k - 1} = r_{k - 2} - q_{k}(r_{k - 3} - q_{k - 1}r_{k - 2}) = \ldots = k_{1}a + k_{2}b$ --- выражение через $a$ и $b$
\end{proof}

\begin{lemma}\label{lemma:2}
  Если $a, b \in \mathbb{Z}, p$ --- простое и $p | ab$, то $p|a$ или $p|b$
\end{lemma}
\begin{proof}
  Пусть $p \nmid a$ и $p \nmid b \Rightarrow (p, a) = 1$ и $(p, b) = 1$\\
  По Лемме~\ref{lemma:1} $\exists u_{1},u_{2},v_{1}, v_{2} \in \mathbb{Z}$\\
  $u_{1}p+v_{1}a = 1$\\
  $u_{2}p+v_{2}b = 1$ \\
  перемножим: $u_{1}u_{2}p^{2} + u_{1}v_{2}pb + u_{2}v_{1}pa + v_{1}v_{2}ab = 1 \Rightarrow p|1$ -- противоречие
\end{proof}

\begin{theorem}[Основная теорема арифметики]
  Любое целое число $a \in \mathbb{Z}$ можно представить в виде $a = \pm p_{1}p_{2}\ldots p_{k}$, где $p_{1}, \ldots, p_{k}$ --- простые числа.
  Такое представление единственно с точностью до порядка множителей.
\end{theorem}
\begin{proof}
  \emph{Существование:}
  $a \in \mathbb{Z}$. Если $a = \pm p$, $p$ --- простое, то $a = \pm p$ --- требуемое разложение.
  Если $a$ --- составное, то $a = bc$, где $|b|, |c| < |a|$.
  Далее применяем те же рассуждения к $b$ и $c$.

  \emph{Единственность:}
  $a = \pm p_{1} \ldots p_{k} = \pm q_{1} \ldots q_{l}$\\
  $p_{1}|q_{1} \ldots q_{l} \Rightarrow$ (по Лемме~\ref{lemma:2}) $\exists i p_{i}|q_{i}, p_{i},q_{i}$ --- простые $\Rightarrow p_{i} = q_{i}$.
  Можно считать, что $i = 1, \Rightarrow p_{2} \ldots p_{k} = q_{2} \ldots q_{l}$, и т.д.
\end{proof}
\end{document}
